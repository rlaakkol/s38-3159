\documentclass[a4paper]{article}

\usepackage[english]{babel}
\usepackage[utf8x]{inputenc}
\usepackage{graphicx}
\usepackage[ELEC]{aaltologo}
\usepackage{verbatim}
\usepackage[bf,small]{caption}
\usepackage[toc,page]{appendix}

\usepackage{pdfpages}

\begin{document}
% Titlepage
\begin{titlepage}

% Upper part of the page
\begin{flushleft}
\AaltoLogoSmall{1}{?}{aaltoYellow} \\
Department of Communications and Networking \\
S-38.3159 Protocol Design P \\
Spring 2014
\end{flushleft}

\vspace{8cm}
% Title
\begin{center}
  {\LARGE \textbf{aNSA is Not a Sensor Aggregator}}\\
  {\LARGE Protocol implementation documentation}
\end{center}

\vfill

\begin{center}
\begin{tabular}{rll}
	Student 1:	& Riku Lääkkölä 	& 69896S \\
				& riku.laakkola@aalto.fi & \\    
    Student 2: 	& Tero Marttila		& 78949E \\
    			& tero.marttila@aalto.fi & \\
    Student 3:	& Tero Paloheimo	& 78510C \\
    			& tero.paloheimo@aalto.fi & \\
\end{tabular}
\end{center}

\end{titlepage}

\tableofcontents
\newpage

\section{Program structure}
The our implementation consists of the binaries: client (ps-client) and server
(ps-server). Both are backwards compatible, supporting protocol revisions for
assignments 1 and 2. The server maintains state for subscribed clients and
available sensors. The client sends subscribe queries and receives \emph{publish}
messages sent by the server.

A shared library is used by both binaries. The library contains classes for
the client, server, sensors, parsing and data transmission.

\section{Implementation status}
The header format and messages types are implemented by the specification.
The sequence number wrap around is not implemented. The subscribe-query is done
as specified by the specification but subscription update and unsubscribing
are not implemented.

Publishing is done according to the specification, but the keep-alive (NoAck
unset) publish messages from the server should have contained the values from
the subscribed device sensors.

% Header format:
%     No changes
% Message types:
%     No changes
% Message exchange:
%     Subscribe-query:
%         No changes
%     Subscribe:
%         No changes, updating subscription and unsubscribing not implemented.
%     Publish:
%         According to the spec, the keep-alive (NoAck unset) publish messages from the server should have contained the values from the subscribed device sensors 

\section{Packet loss testing}

\section{Testing scenario Z}
We tested the ``scenario Z'' by creating a shell script
for starting the sensors and then running one server and one client. If all
sensors where running, the client was unable receive any messages from the
server causing the server to time out. When one camera and audio sensor along
with the other sensor types where running, the sensor received the \emph{publish}
messages correctly. Thus we can conclude that running too many (at least five
camera and seven audio sensors) simultaneous camera and audio sensors consume
all of the server's processing time.

\section{Lessons learnt}
We would likely have done most things the same way if we had to the assignment
again from scratch but naturally some things would have been changed. The most
common thing was that certain values or semantics in the specification were
missing or unclear. One example is the retransmission timeout for \emph{teardown}
messages sent by the client. Also the semantics for the aggregation methods
were unspecified and some aggregation methods for the camera and audio sensors
could have been added. 

% Design:
% - Specify default values for all timeouts, particularly teardown
%   - Can the timeouts be changed?
% - Specify semantics for aggregation expressions
% - Add aggregation expressions also for other sensors, e.g. camera and asd

\section{Learning diary}
The learning diary consists mostly of matters relating to the design and the
assignment 1 implementation of the protocol. The diary is shown below.

The detailed implementation progress can be be seen from the commit log of our
Git repository. It is included in Appendix \ref{lst:commit_log}.

\input{AssignmentDiary.tex}

\clearpage
\begin{appendices}
  \section{Git commit log}\label{lst:commit_log}
  \verbatiminput{commit_log.txt}
\end{appendices}

\end{document}
